\documentclass{article}
\begin{document}
\title{Hit Detection}
\author{31core}
\maketitle
\newpage
%----- Catalog -----
\tableofcontents
\newpage
\setlength{\parindent}{0em}
\section{Vector expression of Ray}
Any ray can be expressed using a point and a unit vector like this:
\begin{equation}
R = A + t\vec{B}
\end{equation}

\section{Hit detection of Sphere}
\subsection{Mathematical proof}
Because $\vec{CA}$ is perpendicular to $\vec{B}$, so we can get this equation.

$(\vec{CA} + t_d\vec{B})\vec{B} = 0$

We can get $t_d$ like this:

$t_d = \frac{ - \vec{CA} \cdot \vec{B} }{ \vec{B}^2 } = -\vec{CA} \cdot  \vec{B}$

Then calculate the distance from the center of the sphere to the ray:

$d = |\vec{CA} + t_d\vec{B}|$

If $d \le R$, the ray do hit the sphere.

Finally, calculate the $t$.

$t = \frac{|t_d\vec{B}| - \sqrt{R^2 - d^2}}{|\vec{B}|} = t_d - \sqrt{R^2 - d^2}$

\subsection{Final formula}
\begin{equation}
t_d = -\vec{CA} \cdot  \vec{B}
\end{equation}
\begin{equation}
d = |\vec{CA} + t_d\vec{B}|
\end{equation}
\begin{equation}
t = t_d - \sqrt{R^2 - d^2}
\end{equation}

\section{Reflection of Ray}
\subsection{Mathematical proof}
$-\vec{A} + \vec{B} = t\vec{N}$

$(\frac{t}{2}\vec{B} + \vec{A})\vec{N} = 0$

$t = -\frac{2\vec{A}^2}{\vec{N} \cdot \vec{A}} = -\frac{2}{\vec{N} \cdot \vec{A}}$

\subsection{Final formula}
\begin{equation}
t = -\frac{2}{\vec{N} \cdot \vec{A}}
\end{equation}
\begin{equation}
\vec{B} = t\vec{N} + \vec{A}
\end{equation}

\end{document}
